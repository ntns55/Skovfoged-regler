\chapter{Indledning}

Denne profession \textbf{kræver} special ansøgning. \\
Som Tempelkriger kan du bruge alle typer rustning og alle typer våben der ikke kræver specialansøgning.\\
Du kan maksimalt få 8 RP fra rustning, uanset hvor meget rustning du har. Det vil sige at selv hvis du får at vide at du har 13 RP ved tjek ind, så har du kun 8.\\

Som Tempelkriger er din opgave at være det sværd din tro forlanger. Selv den mest passifistiske gud har brug for nogen der kan stå mellem uskyldige og vold. Som tempelkriger behøver du ikke tro på en gud, du kan også være dedikeret til naturen, men uanset hvad så giver de dig magt du ikke normalt ville have adgang til.

\chapter{At kaste magi som en Tempelkriger}
\input{../Evne-Ordbog/Generelt om Magi.tex}

Man kaster en magi ved at sige en bøn. Denne bøn skal minimum være 5 ord for hvert niveau den ønskede magi er. (Med undtagelse af niveau 1 som starter på 10 ord.) Hvis man siger det samme ord 2 gange tæller det stadig kun som 1 ord.\\
\begin{table}[H]
    \centering
    \begin{tabular}{c|c}
        Niveau magi & Antal ord i bøn \\\hline
        1 & 10\\
        2 & 15\\
        3 & 20\\
        4 & 25\\
    \end{tabular}
\end{table}

Hvis man bliver afbrudt i ens bøn, skal man starte helt forfra. Magien vil ikke blive brugt og derfor koster forsøget ikke noget mana.\\
\textit{Eksempel: Jeg vil kaste en niveau 2 magi. Derfor siger jeg en bøn på mindst 15 ord, og da jeg siger ”store” to gange skal jeg mindst sige 16 ord. Man må godt sige mere end de nødvendige ord per niveau, dette er kun et minimum.}

\section{Altar}
En Tempelkriger har et sted som et helligt for deres tro. Dette sted refereres til som tempelkriger altar. Et altar skal have et sted hvor der kan bedes en bøn og det skal som minimum have din tro's ikon.

\section{Genvinde mana som tempelkriger}
Man kan regenerere ens mana ved at holde en messe eller prædike for en forsamling. For hvert minut en tempelkriger messer eller prædiker får præsten 1 mana.
